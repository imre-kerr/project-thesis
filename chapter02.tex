% !TEX encoding = UTF-8 Unicode
%!TEX root = thesis.tex
% !TEX spellcheck = en-US
%%=========================================
\chapter{Previous work}
What follows is a brief treatment of previous work that has been done on the Snow Simulator. Only the aspects that are relevant to my project are included. For a fuller treatment, I refer to any of a number of recent masters theses about the subject, e.g. \citet{boge2014avalanche}.
%%=========================================
\subsubsection*{History}
The Snow Simulator started out as a Master's project by \citet{saltvik2006parallel}. It was CPU only, and could simulate 40 000 snow particles in real time. Later, it was ported to CUDA by \citet{eidissen2009utilizing}, which (along with hardware advances, of course) brought the particle count up to 2 000 000.

After this, the simulator has been the subject of several master's theses, which have all improved or expanded on it in some way. Examples include visualization improvements \citep{nordahl2013enhancing}, avalanche prediction \citep{boge2014avalanche}, and porting to OpenACC \citep{mikalsen2013openacc} and OpenCL \citep{vestre2012enhancing}.
%%=========================================
\subsubsection*{Architecture}
Initially, the user is greeted with a screen that allows setting various options such as number of particles, wind field size, terrain map file, whether to simulate wind and so forth. When these settings have been entered, the simulation can be started by the user.

The simulator starts by initializing the data structures needed for the simulation. A heightmap is generated based on the terrain map file, the wind field is initialized (no wind initially), and randomized positions for snow particles are generated. After initialization, the simulator enters the main loop, which runs through the various simulation steps on the GPU, before rendering everything to the screen. The main simulation steps are:
\begin{enumerate}
\item Update obstacle field
\item Simulate wind
\item Move snow particles
\item Update snow layers
\end{enumerate}

The GPU specific code is kept quite separate from the rest of the simulator. The reason for this is that most of the simulator is written in C++, but CUDA programs must be compiled with nvcc, which only accepts C. Therefore, the CUDA code is compiled into a library, libParticle.a, which is then statically linked to the rest of the simulator. All communication with this library is done using a relatively small set of public methods.

Internally, the particle system library is divided into three subsystems: Snow, Wind and Terrain. Data passes quite freely between these three, mostly using global state. 

%%=========================================
\subsubsection*{Existing OpenCL port}
As mentioned, there already exists an OpenCL port of the Snow Simulator, written by \citet{vestre2012enhancing}. However, this port has become out of date due to not being updated along with the CUDA version. Hence, this project is an attempt to bring it up to date. When planning the project, two approaches were considered: 
\begin{itemize}
\item Base my work on the previous port, merging in all the improvements that have happened over the past two years.
\item Base my work on the most recent version. This effectively means redoing the porting work, but will entail less work due to the possibility of copying code in areas where the simulator is unchanged.
\end{itemize}
I decided on the second approach, not only because it meant less work, but also because the work would be localized to only one part of the simulator, rather than spread out across the whole codebase.